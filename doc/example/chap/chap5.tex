\chapter{重要文献与研究团队总结}

\section{重要文献总结}
\textbf{1. }Li M, Andersen D G, Park J W, et al. Scaling distributed machine learning with the parameter server[C]//11th {USENIX} Symposium on Operating Systems Design and Implementation ({OSDI} 14). 2014: 583-598.

\textbf{2. }Harlap A, Tumanov A, Chung A, et al. Proteus: agile ml elasticity through tiered reliability in dynamic resource markets[C]//Proceedings of the Twelfth European Conference on Computer Systems. 2017: 589-604. \textbf{\&} Harlap A, Chung A, Tumanov A, et al. Tributary: spot-dancing for elastic services with latency SLOs[C]//2018 {USENIX} Annual Technical Conference ({USENIX}{ATC} 18). 2018: 1-14.

\textbf{3. }Crankshaw D, Wang X, Zhou G, et al. Clipper: A low-latency online prediction serving system[C]//14th {USENIX} Symposium on Networked Systems Design and Implementation ({NSDI} 17). 2017: 613-627.

\textbf{4. }Zhang C, Yu M, Wang W, et al. Mark: Exploiting cloud services for cost-effective, slo-aware machine learning inference serving[C]//2019 {USENIX} Annual Technical Conference ({USENIX}{ATC} 19). 2019: 1049-1062.

\textbf{5. }Amaral M, Polo J, Carrera D, et al. Topology-aware gpu scheduling for learning workloads in cloud environments[C]//Proceedings of the International Conference for High Performance Computing, Networking, Storage and Analysis. 2017: 1-12.

\textbf{6. }Xiao W, Bhardwaj R, Ramjee R, et al. Gandiva: Introspective cluster scheduling for deep learning[C]//13th {USENIX} Symposium on Operating Systems Design and Implementation ({OSDI} 18). 2018: 595-610.

\textbf{7. }Vaucher S, Pires R, Felber P, et al. SGX-aware container orchestration for heterogeneous clusters[C]//2018 IEEE 38th International Conference on Distributed Computing Systems (ICDCS). IEEE, 2018: 730-741.

\textbf{8. }Venkataraman S, Yang Z, Franklin M, et al. Ernest: Efficient performance prediction for large-scale advanced analytics[C]//13th {USENIX} Symposium on Networked Systems Design and Implementation ({NSDI} 16). 2016: 363-378.

\textbf{9. }Alipourfard O, Liu H H, Chen J, et al. Cherrypick: Adaptively unearthing the best cloud configurations for big data analytics[C]//14th {USENIX} Symposium on Networked Systems Design and Implementation ({NSDI} 17). 2017: 469-482.

\textbf{10. }Mao H, Schwarzkopf M, Venkatakrishnan S B, et al. Learning scheduling algorithms for data processing clusters[M]//Proceedings of the ACM Special Interest Group on Data Communication. 2019: 270-288.


\section{重要研究团队总结}
\textbf{Ion Stoica团队。}University of California, Berkeley的Ion Stoica教授团队在大数据处理、数据中心资源管理以及人工智能等领域取得了丰硕的研究成果。在数据处理方面,该团队提出开发了Spark、Shark、SparkR、GraphX等项目,不仅在学术界产生了影响力,并且在工业界得到了广泛利用。在这些研究成果的基础上,该团队的核心成员还创立以数据处理技术为核心的公司Databricks, 目前估值已经达到了27.5亿美元。在数据中心资源管理方面,该团队提出了DRF、Delay Scheduling等调度策略,在真实的集群调度器中得到了应用。同时,该团队也是双层调度系统Mesos与分布式调度系统Sparrow的提出者。在云资源预测与配置方面,该团队提出了Ernest。近年来,该团队也开始涉足机器学习领域,研究支撑机器学的系统和框架开发,提出了面向机器学习的分布式框架Ray。

\textbf{Christos Kozyrakis团队。}Stanford University的Christos Kozyrakis教授团队在云资源管理领域上取得了丰富的研究成果。该团队在2013年时提出的Paragon,在2014年提出的Quasar都将基于机器学习的分类技术应用到了集群调度系统中,在保证应用QoS的基础上,提高了集群的资源利用率。在2015年,该团队又提出了Tarcil和Heracles。Tarcil是一个分布式调度器。Heracles则将延迟敏感型作业与批处理型作业运行在一起来提升集群资源利用率。在2016年,该团队提出了高效的云资源分配系统HCloud。在2017年,该团队提出了面向云安全的Bolt系统。在2018年,该团队提出了面向数据处理作业的云计算和存储资源优化配置的工具Selecta。