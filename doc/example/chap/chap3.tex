% Copyright (c) 2014,2016,2018 Casper Ti. Vector
% Public domain.

\chapter{云环境中的异构硬件为人工智能带来的机遇与挑战}
% \pkuthssffaq % 中文测试文字。
随着硬件技术的不断发展,云环境中支持机器学习任务的硬件也从单一的CPU变成了包含CPU、GPU、FPGA、TPU以及SGX等安全硬件的异构资源。利用这些异构资源,可以给机器学习任务带来一定增益。例如GPU,TPU和FPGA等硬件的SIMD计算模式天然适合ML这种迭代式、大批量的负载,从而可以大大加速ML训练。而诸如SGX等硬件由让机器学习安全有了从硬件层面得到解决的可能性。

而异构硬件为ML任务带来机遇的同时,也带来了一定的挑战。例如,在公有云共享资源池的背景下,如GPU和FPGA等新兴加速硬件没有成熟和系统的虚拟化方案,用户的使用这些硬件的模式仍然以独占为主。即便是共享,也没有成熟的隔离机制保证性能以及用户之间不会相互干扰。再比如,SGX加持的安全计算环境中,可信内存的大小十分有限,如何在有限的资源下支持可信的模型部署甚至模型训练,是亟待解决的问题。

本章将以加速型硬件和安全性硬件为主,介绍云环境中异构硬件为机器学习任务带来的机遇与挑战。

\section{}
% vim:ts=4:sw=4
