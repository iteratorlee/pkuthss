\chapter{下一步的研究设想}

\section{研究动机}
随着云商各种新硬件和新型计算模式的不断推出,用户所面临的云上资源池越来越复杂,为应用程序配置最优的资源也变得越来越困难。而现有的作业性能建模与配置优化技术,多关注于传统的CPU资源和IaaS服务,对新型硬件(例如GPU、SGX等)和新型服务模式(如Serverless Computing)上的作业性能建模和配置优化关注不足。同时针对传统CPU资源的性能建模方法仍然存在若干缺陷和不足,例如:1)预运行成本仍然较高,对于某些长作业来讲预运行一次付出的代价是不可接受的;2)先前的工作对运行时信息的重视程度不够,要么没有将其纳入模型中,要么将其视作静态的参数,而非一个长度可变的时间序列。

因此,作者认为,在新硬件和新服务模式日新月异的今天,云环境作业配置优化技术仍然有非常大的研究空间。基于此动机,作者提出如下研究计划。

\section{研究计划}
作者的研究计划分为基础方法和系统两部分。在基础方法部分,作者计划对CPU、GPU和SGX三类资源上的作业配置优化和性能建模做系统性的研究,提出新的方法或超出现有工作的方法。在系统方面,作者计划以机器学习作业(ML Job,包含ML工作流中的训练和部署两个步骤)为例,构建一个在云上自动为ML Job配置资源、自动部署Job、根据云资源的变化情况自动调整资源的系统。

\textbf{针对云上的异构服务和异构硬件,形成一套完整的作业性能建模和作业配置优化方法。}
\begin{enumerate}
    \item 针对仅使用到CPU资源的作业,提出一种预运行次数更低、性能更优的配置优化方法。
    \item 针对使用到GPU资源的作业,提出一种GPU环境下作业性能预测、性能干扰预测的方法。
    \item 针对安全型作业,提出一种SGX环境下作业性能预测、性能干扰预测的方法。
\end{enumerate}

\textbf{以上述方法为基础,形成一个自动为机器学习作业配置并动态调整资源的系统。}
\begin{enumerate}
    \item 根据用户不同的需求(例如SLO目标、成本目标、安全性目标或基于上述目标的组合目标等),系统为ML模型训练任务和ML模型部署任务在云上选择最优的服务+硬件组合,并在其上进行作业的自动部署。
    \item 作业部署之后,系统根据云市场资源属性(例如资源性能和资源价格)的动态变化,对作业占用的资源进行动态调整和调度,使其持续满足用户的需求。
\end{enumerate}

%\subsection{结合频域信息对数据处理类任务进行更精确的建模}