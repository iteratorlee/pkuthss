\chapter{下一步的研究设想}

\section{研究动机}
随着云商各种新硬件和新型计算模式的不断推出,用户所面临的云上资源池越来越复杂,为应用程序配置最优的资源也变得越来越困难。而现有的作业性能建模与配置优化技术,多关注于传统的CPU资源和IaaS服务,对新型硬件(例如GPU、SGX等)和新型服务模式(如Serverless Computing)上的作业性能建模和配置优化关注不足。

因此,作者认为,在新硬件和新服务模式日新月异的今天,云环境作业配置优化技术仍然有非常大的研究空间。基于此动机,作者提出如下研究计划。

\section{研究计划}

\textbf{针对云上的异构服务和异构硬件,形成一套完整的作业性能建模和作业配置优化方法。}
\begin{enumerate}
    \item \textbf{基于Job预运行中监测得到的系统指标(例如CPU利用率、内存利用率、换页频率、中断次数等),建立一个Job的性能预测基础模型。}当前存在两类Job的性能预测/配置优化方法,分别是基于线上搜索的方法和基于线下建模的方法。前者在workload运行时间较长时预运行的overhead过大,而后者在作业描述等方面仍然存在缺陷。作者拟提出一种基于线下建模的性能预测/配置优化方法,充分提取底层监控指标中的信息,完善对作业的描述,使模型达到更高的性能。同时,在预运行时利用小规模数据输入采集作业信息,尽可能降低用户预运行的overhead。
    \item \textbf{基于Job的特征(例如Job的资源利用曲线),建立Job之间的性能干扰基础模型。}当前对Job干扰建模的方法一般是静态的。作者拟根据作业的资源利用曲线,提出一种动态的作业性能干扰模型。例如,两个Job之间的性能干扰度可能会随着时间的变化而变化,且与作业启动时间的相位差相关。
    \item \textbf{在不同的硬件/服务环境下,对上述两个基础模型进行修正。}在不同的硬件环境或者服务环境中,可能存在一些特有的特征。例如,GPU任务和CPU任务在底层指标上存在差异,SGX中无法对Job的EPC利用率进行直接观测等等。因此,需要针对不同的环境,为模型设计不同的特征。
    \item \textbf{基于上述两个基础模型,以机器学习Job(包括训练和部署)为例,自动为其在云上配置最优的资源。}基于上述两个模型,根据用户不同的需求(例如SLO目标、成本目标、安全性目标或基于上述目标的组合目标等),为ML模型训练任务和ML模型部署任务在云上选择最优的服务+硬件组合,并在其上进行作业的自动部署。
\end{enumerate}

%\subsection{结合频域信息对数据处理类任务进行更精确的建模}