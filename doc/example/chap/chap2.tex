% Copyright (c) 2014,2016 Casper Ti. Vector
% Public domain.

\chapter{面向人工智能的云计算系统软件研究}
% \pkuthssffaq % 中文测试文字。
以深度学习为代表性技术的人工智能领域在21世纪10年代再度兴起,社会各界对于人工智能的需求也愈发旺盛。遍布超市和餐馆中的扫脸支付技术、智能手机上的语音智能助手、工厂中检测废件的自动检测装置等,都离不开人工智能技术的加持。具体的,上述场景都需要适当的机器学习模型在线部署以提供服务。

机器学习的工作流一般遵循如下几个步骤:1)模型开发与调试。在此阶段中,开发者根据具体的场景,编写并调试模型代码。2)模型参数调整。在这个步骤中,开发者使用训练集不断调整模型的超参数,使其在测试集上的表现符合某种标准(如准确率高于某个阈值)。3)模型部署,开发者将开发完成的模型部署到对应的场景中对外提供服务。

为了方便开发者专注到模型的开发流程中,各大主流云厂商均实现了支持机器学习模型开发-调试-部署整个流程的软件栈。同时,云厂商提供的一般性的平台可能无法满足用户特定的需求。因此针对具体场景,学术界也提出了一系列基于云服务的机器学习软件,以达到降低开发成本、提高模型在线服务的质量等目标。本章对上述内容涉及到的相关工作展开具体研究。

\section{一站式的云上机器学习系统}

本节先以AWS Sagemaker为代表,介绍公有云厂商一站式的云上机器学习系统。其它厂商中类似系统中对计算模型的抽象、使用方式等均与Sagemaker大同小异。

\subsection{AWS Sagemaker}
AWS作为公有云领域的先驱者,在云计算的前沿技术领域一直处于领先的地位。其某些代表性技术甚至成为了多数公有云厂商所共识的标杆和规范。在机器学习系统这一分领域,其代表性系统为AWS Sagemaker。

Sagemaker是一个实现和产品化机器学习模型的框架\parencite{joshi2020amazon},用户可以将其模型训练需要的数据存储到AWS的对象存储服务S3中。然后,用户可以通过web的方式访问部署在云端的Jupyter Notebook服务,在线访问数据、编写并调试代码。当模型训练完毕后,可以使用Sagamaker内置的推理服务对模型进行发布。用户在发布时还可以根据平台的不同(云端或边缘节点)将模型打包编译成不同的版本。此外,Sagamaker还内置了一些常用的算法和数据集,方便开发者直接调用。

在训练算法和系统机制方面,Sagemaker旨在解决工业规模的模型训练场景中的如下几个常见问题:

\begin{itemize}
    \item \textbf{支持增量式的训练和模型更新。}在真实的工业场景中,几乎不存在完全静态的数据集。用来训练模型的数据集大多都是不断增长的。例如电商网站的用户行为数据,每天都在以相当的速度增长。在这样的动态数据上训练模型,势必要进行如下权衡:在全量数据上进行训练,可以获得质量更高的模型,然而时间和经济上的开销却会非常高;在最新更新的数据上(例如最近几天的数据)进行训练,可以快速的得到新的模型,却有可能在一定程度上牺牲模型的准确性。
    \item \textbf{容易估算训练模型产生的花销。}对于体量非常大的数据集,用户需要较为准确地估计训练模型将会产生的时间和经济花销。当今的云计算一般遵循按量付费的收费模式,因此云上的机器学习用户会格外关注花销的问题。
    \item \textbf{支持暂停和恢复模型训练,有一定的弹性。}生产环境中,大模型的训练通常包含跨越数十甚至上百台机器的并发任务。在一些场景中,由于超参数调整的需要或者计算资源的限制,开发者需要对这些任务进行中断和恢复。这就要求云上的ML系统能够支持大型训练任务中断和恢复时中间结果的保存和复原。
    \item \textbf{能够处理非持久性数据。}在很多场景中,数据并不一定是持久化的,也会有很多“瞬时”的数据,例如直播时的视频流等。这些数据一般不会被持久化保存,因此如何支持对这些数据的挖掘和学习也是一个重要的问题。
    \item \textbf{支持自动调参。}自动调参是一项非常耗时耗力的工作,特别是在生产环境的大数据集下。因此,如何能够支持高效的自动调参,方便用户选取合适的模型,对于云上的ML系统而言也是非常重要的。
\end{itemize}

下面将从计算模型、支持算法和具体的实现方式介绍Sagemaker如何解决上述问题。

\subsubsection{计算模型}
Sagemaker将机器学习训练的工作流抽象为三个阶段:initialize,update和finalize。initialize阶段:对相关的变量进行初始化。update阶段:系统以数据流及其状态作为输入,根据用户编写算法更新状态。在Sagemaker的后台,状态更新是在多台计算节点上同步执行的。finalize阶段:对之前所有的更新进行汇总,并输出最后的结果(对于机器学习任务而言,一般是一个模型)。通过上述抽象,用户只需编写initialize、update和finalize三个函数,就可以将整个训练过程交由Sagemaker,等待最后的模型输出。

图\ref{sage_cmp_mdl}用伪代码的形式讲述了三个阶段的具体流程。这里以计算一列数据的中位数为例,使用随机梯度下降(SGD)进行求解。对于如下一列数据:$x_1, x_2, x_3, ..., x_n$,则由如下公式给出这一列数据的中位数:$\operatorname{argmin}_z\Sigma_{i=1}^n||x_i-z||$。算法中的initialize函数首先对median和n两个参数做初始化。这里\textit{state}可以理解为一个key-value存储结构。在update阶段,函数将接收到的数据流\textit{data\_stream}分为若干mini\_batch,对每个batch没的数据,迭代式的对median值进行更新,同时更新n的值。最后,在finalize函数中,对最终的median值进行汇总。同时,系统还支持用户在算法的任意位置设置barrier(第15行),以对整个训练过程进行同步操作。

\begin{figure}
    \centerline{\includegraphics[width=\textwidth]{figures/compute-model.png}}
    \caption{Sagemaker计算模型}
    \label{sage_cmp_mdl}
\end{figure}

\subsubsection{支持算法}
\subsubsection{实现方式}

\subsection{其它厂商的相关系统简介}

\section{基于公有云服务的模型训练系统}
尽管公有云厂商实现了相当完备的一站式机器学习系统
\subsection{基于云厂商动态资源的模型训练系统}
% harlap的第一篇工作

\subsection{基于云厂商动态资源的自动调参系统}
% SpotTune

\section{基于公有云服务的模型部署系统}

\subsection{基于多种资源混用的模型部署系统}
% MArk

\subsection{基于云厂商动态资源的服务部署系统}
% Tributary

\subsection{基于FaaS的模型部署系统}

%一系列FaaS for ML serving的工作,INFaaS etc
% vim:ts=4:sw=4
