% Copyright (c) 2014,2016 Casper Ti. Vector
% Public domain.

\chapter{面向人工智能的云计算系统软件研究}
% \pkuthssffaq % 中文测试文字。
以深度学习为代表性技术的人工智能领域在21世纪10年代再度兴起,社会各界对于人工智能的需求也愈发旺盛。遍布超市和餐馆中的扫脸支付技术、智能手机上的语音智能助手、工厂中检测废件的自动检测装置等,都离不开人工智能技术的加持。具体的,上述场景都需要适当的机器学习模型在线部署以提供服务。

机器学习的工作流一般遵循如下几个步骤:1)模型开发与调试。在此阶段中,开发者根据具体的场景,编写并调试模型代码。2)模型参数调整。在这个步骤中,开发者使用训练集不断调整模型的超参数,使其在测试集上的表现符合某种标准(如准确率高于某个阈值)。3)模型部署,开发者将开发完成的模型部署到对应的场景中对外提供服务。

为了方便开发者专注到模型的开发流程中,各大主流云厂商均实现了支持机器学习模型开发-调试-部署整个流程的软件栈。同时,云厂商提供的一般性的平台可能无法满足用户特定的需求。因此针对具体场景,学术界也提出了一系列基于云服务的机器学习软件,以达到降低开发成本、提高模型在线服务的质量等目标。本章对上述内容涉及到的相关工作展开具体研究。

\section{}

\section{}
% vim:ts=4:sw=4
