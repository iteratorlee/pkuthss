% Copyright (c) 2014,2016 Casper Ti. Vector
% Public domain.

\begin{cabstract}
	%\pkuthssffaq % 中文测试文字。
	21世纪10年代以来,云计算和人工智能可谓计算机科学领域最为炙手可热的两个研究方向。以虚拟化、资源管理和服务化为代表的云计算核心技术在近十余年里取得了丰硕的研究成果。当前,云计算已成为工业化社会重要的信息基础设施,支撑并推动着大数据和人工智能产业的快速发展。与此同时,人工智能技术在近十年内也相继在计算机视觉、自然语言处理等多个领域取得了突破,“智能化”已然成为现代社会的重要标签之一。
	
	本文将以上述两大技术的蓬勃发展为背景,研究云计算与人工智能相互影响、相互支持、相辅相成的相关技术。本文将按照如下几章展开。
	
	第一章对相关的技术背景做出介绍。首先对云计算近十年的发展做简要回顾,并介绍具有代表性的若干核心技术。其次对人工智能近十年的发展做简要概述,阐述其在计算机视觉、自然语言处理等子领域的代表性科研成果。最后分析二者在交叉领域现有的相关研究,即利用人工智能技术解决云计算中的资源配置与调度问题,以及云计算环境中在软件和硬件层面对人工智能应用的支撑情况。
	
	第二章开始探究二者的关系。本章从“服务于云计算系统架构的AI技术”这一视角展开,研究人工智能对云计算的增强技术。云计算本质上是一个巨大的公共资源池,用户如何在资源池中选取资源,云厂商如何为不同的用户调度资源,是云计算领域两个重要的话题。近五年来,该领域的研究者开始尝试利用一系列基于机器学习的算法来辅助解决上述两个问题,本章将重点讨论与上述算法相关的研究。
	
	第三章从另一角度,研究云计算环境中用以支持人工智能的系统软件技术。随着人工智能算法和系统研究的发展,其训练-测试-部署的流程愈发复杂。很多云厂商基于本地化的云原生技术,构建了一站式的人工智能开发-部署软件栈供用户使用。同时,伴随着新的云计算模式(如无服务计算serverless)的产生,工业界和学术界也在探究将其应用在人工智能领域,使相关的应用在云上具有更高的弹性。本章将重点讨论与上述话题相关的系统软件研究工作。
	
	第四章讨论近些年来云环境下出现的新硬件,如GPU,AI专用芯片,Intel-SGX等,为人工智能技术带来的新的机遇与挑战。硬件的发展(如GPU,AI专用芯片等)导致的算力的提升,也是人工智能近年发展迅速的重要原因之一。同时,某些硬件层面安全机制(如Intel-SGX)的产生,使增强人工智能应用的安全性有的新的潜在解决方案。本章将关注近十年来的新硬件为人工智能带来的新的机遇与挑战。
	
	第五章总结了上述三个方向的重要文献和相关研究团队概况。
	
	第六章介绍了作者下一步的研究计划。

\end{cabstract}

%\begin{eabstract}
%	Test of the English abstract.
%\end{eabstract}

% vim:ts=4:sw=4
