% Copyright (c) 2014,2016 Casper Ti. Vector
% Public domain.

\begin{cabstract}
	%\pkuthssffaq % 中文测试文字。
	21世纪10年代以来,云计算和人工智能可谓计算机科学领域最为炙手可热的两个研究方向。以虚拟化、资源管理和服务化为代表的云计算核心技术在近十余年里取得了丰硕的研究成果。当前,云计算已成为工业化社会重要的信息基础设施,支撑并推动着大数据和人工智能产业的快速发展。与此同时,人工智能技术在近十年内也相继在计算机视觉、自然语言处理等多个领域取得了突破,“智能化”已然成为现代社会的重要标签之一。
	
	本文将以上述两大领域的蓬勃发展为背景,研究云计算与人工智能相互影响、相互支持、相辅相成的相关技术。本文将按照如下几章展开。
	
	第一章对相关的技术背景做出介绍。首先对云计算近十年的发展做简要回顾,并介绍具有代表性的若干核心技术。其次简要概述人工智能的发展历史,及其近十年在计算机视觉、自然语言处理等领域的代表性成果。最后分析二者在交叉领域现有的相关研究。
	
	第二章开始探究二者的关系,研究云计算环境中用以支持人工智能的系统软件技术。随着人工智能算法和系统研究的发展,其训练-测试-部署的流程愈发复杂。主流的公有云厂商基于本地化的云原生技术,构建了一站式的机器学习模型开发-部署软件栈供用户使用。然而云厂商提供的大而全的一站式机器学习系统很难覆盖所有的应用场景。在一些成本敏感的、性能敏感的应用场景中,部分学者也在尝试利用公有云提供的服务搭建第三方的机器学习系统,以实现特异性的需求。同时,随着以Serverless Computing为代表的新兴云计算模式的出现,部分研究者也在探索如何将其应用在机器学习系统的构建上,使机器学习应用在云上具有更高的弹性。

	第三章讨论云环境中的异构硬件(如GPU,AI专用芯片,Intel SGX等)为人工智能技术带来的新的机遇与挑战。以GPU为代表的加速型硬件的发展导致的算力的提升,也是人工智能近年发展迅速的重要原因之一。如何使机器学习任务在这类加速硬件上更快的运行,并提升集群的资源利用率,是学术界关注的重要问题。同时,以Intel SGX为代表的可信执行环境技术的兴起和发展,为增强人工智能应用的安全性提供了新的解决方案。

	第四章从另一角度,即“服务于云计算系统架构的AI技术”这一视角展开,研究人工智能对云计算的增强技术。云计算本质上是一个巨大的公共资源池,用户如何在资源池中选取资源,云厂商如何为不同用户的任务调度资源,是云计算领域的两个重要话题。近五年来,部分研究者开始尝试利用一系列基于机器学习的算法来辅助解决上述两个问题。
	
	第五章总结了上述三个方向的重要文献和相关研究团队概况。
	
	第六章介绍了作者下一步的研究计划。

\end{cabstract}

%\begin{eabstract}
%	Test of the English abstract.
%\end{eabstract}

% vim:ts=4:sw=4
